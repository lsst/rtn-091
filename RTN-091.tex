\documentclass[preprint,12pt]{elsarticle}

%% for a journal layout:
%% \documentclass[final,1p,times]{elsarticle}
%% \documentclass[final,1p,times,twocolumn]{elsarticle}

\usepackage{orcidlink}

\usepackage{amssymb}
\usepackage{amsmath}
\usepackage{hyperref}
\usepackage{longtable}

%% \begin{linenumbers}, end it with \end{linenumbers}. Or switch it on
%% for the whole article with \linenumbers.
%% \usepackage{lineno}

\journal{Astronomy and Computing}

% Local commands go here.
\newcommand{\docRef}{RTN-091}
\newcommand{\docUpstreamLocation}{\url{https://github.com/lsst/rtn-091}}

\providecommand{\secref}[1]{Section~\ref{#1}}
\providecommand{\appref}[1]{Appendix~\ref{#1}}
\providecommand{\tabref}[1]{Table~\ref{#1}}
\providecommand{\figref}[1]{Figure~\ref{#1}}
\providecommand{\eqnref}[1]{Eq.~\ref{#1}}
\providecommand{\recref}[1]{REC-\ref{#1}}
\def\VRO{Vera C. Rubin Observatory~}
\def\RO{Rubin Observatory~}
\def\aaps{A\&AS}           % Astronomy and Astrophysics Suplement
\def\aap{A\&A}             % Astronomy and Astrophysics
\def\ssr{Space~Sci.~Rev.}  % Space Science Reviews
\def\apj{ApJ}              % Astrophysical Journal
\def\apjs{ApJS}            % Astrophysical Journal Supplement
\def\aj{AJ}                % Astronomical Journal
\def\mnras{MNRAS}          % Monthly Notices of the RAS
\def\araa{ARA\&A}          % Annual Review of Astron and Astrophys
\def\nat{Nature}           % Nature
\def\apjl{ApJ}             % Astrophysical Journal, Letters
\def\icarus{Icarus}        % Icarus
\def\prd{Phys.~Rev.~D}     % Physical Review D
\def\physrep{Phys.~Rep.}   % Physics Reports
\def\pasp{PASP}            % Publications of the Astronomical Society of the Pacific
\def\procspie{Proc.\ SPIE} % Proceedings of the SPIE
\newcommand{\pasa}{PASA}   % Publications of the Astronomical Society of Australia
\newcommand{\ao}{Appl.~Opt.}  % Applied Optics
\def\pasj{PASJ}            % Publications of the Astronomical Society of Japan

\begin{document}

\begin{frontmatter}

%% Title, authors and addresses

\input{authors}
\date{\today}
\title{Rubin at the Summit - observatory cyber infrastructure and security}

% This can write metadata into the PDF.
% Update keywords and author information as necessary.
\hypersetup{
    pdftitle={Rubin at the Summit - observatory cyber infrastructure and security},
    pdfauthor={silvac},
    pdfkeywords={}
}


\begin{abstract}
We describe the modern and dynamic infrastructure as code based system underlying operations of the Vera. C Rubin Observatory. We will also cover or extra efforts to be NIST standard compliant for cyber security.
\end{abstract}


%%Graphical abstract
%\begin{graphicalabstract}
%\includegraphics{grabs}
%\end{graphicalabstract}


\begin{keyword}
Vera C. Rubin Observatory \sep cybersecurity \sep
%% keywords here, in the form: keyword \sep keyword

%% PACS codes here, in the form: \PACS code \sep code
%% or \MSC[2008] code \sep code (2000 is the default)

\end{keyword}

\end{frontmatter}


\section{Introduction}

The Vera C. Rubin Observatory\cite{2019ApJ...873..111I} will go in to operations in 2025.
During commissioning we have already seen our infrastructure as code based cyber system working well.
In this paper we will describe the generic underlying hardware, deployments on that hardware of both bare metal and containerized applications, transmission of data to SLAC and our NIST\cite{NIST.SP.800-171r3} compliant security approach.






%% Modify acknowldgments as needed.
\textbf{Acknowledgments}\\
This material is based upon work supported in part by the National Science Foundation through Cooperative Agreement AST-1258333 and Cooperative Support Agreement AST-1202910 managed by the Association of Universities for Research in Astronomy (AURA), and the Department of Energy under Contract No. DE-AC02-76SF00515 with the SLAC National Accelerator Laboratory managed by Stanford University.
Additional Rubin Observatory funding comes from private donations, grants to universities, and in-kind support from LSST-DA Institutional Members.




% Include all the relevant bib files so that references can be found from
% lsst-texmf.
% https://lsst-texmf.lsst.io/lsstdoc.html#bibliographies
\bibliographystyle{elsarticle-num}
\bibliography{local,lsst,lsst-dm,refs_ads,refs,books}
\appendix
\input{acronyms.tex}

\end{document}
